\section{Game Theory}

\subsection{Impartial Games}

To be considered a impartial game following rules must be true: 

\begin{enumerate}
    \item The available moves win/loose depends only on the state of the game, in other words, the only difference between the two players is that one of them moves first
    \item Additionally, we assume that the game has perfect information, i.e. no information is hidden from the players (they know the rules and the possible moves).
    \item It is assumed that the game is finite, i.e. after a certain number of moves, one of the players will end up in a losing position — from which they can't move to another position. On the other side, the player who set up this position for the opponent wins. Understandably, there are no draws in this game.
\end{enumerate}

Such games can be completely described by a directed acyclic graph: the vertices are game states and the edges are transitions (moves). A vertex without outgoing edges is a losing vertex (a player who must make a move from this vertex loses).

Since there are no draws, we can classify all game states as either winning or losing. Winning states are those from which there is a move that causes inevitable defeat of the other player, even with their best response. Losing states are those from which all moves lead to winning states for the other player. Summarizing, a state is winning if there is at least one transition to a losing state and is losing if there isn't at least one transition to a losing state.

Our task is to classify the states of a given game.


\subsection{Sprague-Grundy Theorem}
The Sprague-Grundy Theorem states that every impartial game is equivalent to a pile of a certain size in Nim. In other words, every impartial game can be solved as Nim by finding their corresponding game.

Basically, for a game situation $A$ and its $SG$ function value $g(A)$:
\begin{enumerate}
        \item  $g(A)=0$ if and only if A is a must-lose situation. Otherwise, $g(A) \in \Z∗$
        \item If $A$ can be divided into $n$ sub-situations $x_1,x_2,…,x_n$, then $g(A)=g(x_1) \oplus (x_2)\oplus...\oplus g(x_n)$
        \item If $A$ can be converted to situation $B_1$ or $B_2$ or ... or $B_n$ by only one operation, then $g(A)=mex(g(B_1),g(B_2),...,g(B_n))$ where function $mex(S)$ is defined as the smallest non-negative integer that does not appear in $S$. For example, $mex({0,1,2,4})=3,mex{}=0, mex({0,1,2,4})=3,mex({})=0$
\end{enumerate}

\subsection{Nim variation Subtract game}

Work just like nim but instead remove any number of  objects you can remove at most K, this game can be seen as a nim game but before computing the num-sum you need to take the size of each pile module $K+1$, the optimal way to play it is by taking $K$ at each turn.
