
\subsection{Triangle}

Let the lenght of the sides of the triangle be $a$, $b$, $c$.

\subsubsection{Semiperimeter}

Let $p$ be the semiperimeter definded as: 

\begin{equation}
  p = \frac{a + b + c}{2}
\end{equation}

\subsubsection{Area}

Let $A$ be the area defined as:

\begin{equation}
  \sqrt{p(p-q)(p-b)(p-c)}
\end{equation}

\subsubsection{Circumradius}

\begin{equation}
  R = \frac{abc}{4A}
\end{equation}

\subsubsection{Inradius}

\begin{equation}
  r = \frac{A}{p}
\end{equation}

\subsubsection{Lenght of bisector}

\begin{equation}
s_a=\sqrt{bc\left[1-\left(\dfrac{a}{b+c}\right)^2\right]}
\end{equation}

\subsubsection{Law of sines}

\begin{equation}
  \dfrac{\sin\alpha}{a}=\dfrac{\sin\beta}{b}=\dfrac{\sin\gamma}{c}=\dfrac{1}{2R}
\end{equation}

\subsubsection{Law of cosines}

\begin{equation}
  a^2=b^2+c^2-2bc\cos\alpha
\end{equation}

\subsubsection{Law of tangents}

\begin{equation}
  \dfrac{a+b}{a-b}=\dfrac{\tan\dfrac{\alpha+\beta}{2}}{\tan\dfrac{\alpha-\beta}{2}}
\end{equation}


