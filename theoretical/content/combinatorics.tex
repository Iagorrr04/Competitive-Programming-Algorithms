\section{Combinatorics}

\subsection{Binomial Coefficients}

Binomial coefficients $\binom n k$ are the number of ways to select a set of $k$ elements from $n$ different elements without taking into account the order of arrangement of these elements (i.e., the number of unordered sets).

Binomial coefficients are also the coefficients in the expansion of $(a + b) ^ n$ (so-called binomial theorem):

$$ (a+b)^n = \binom n 0 a^n + \binom n 1 a^{n-1} b + \binom n 2 a^{n-2} b^2 + \cdots + \binom n k a^{n-k} b^k + \cdots + \binom n n b^n $$

$$ \binom n k = \frac {n!} {k!(n-k)!} $$

\textbf{Recurrence} formula** (which is associated with the famous "Pascal's Triangle"):

$$ \binom n k = \binom {n-1} {k-1} + \binom {n-1} k $$

\subsubsection{Odd numbers in the i-th line}

O número de elementos ímpares na n-śeima linha do triangulo de pascal é $2^{c}$, onde $c$ é o número de bits
na representação binária de $n$.

\subsubsection{Properties}

Binomial coefficients have many different properties. Here are the simplest of them:


\begin{itemize}
    \item Symmetry rule:

    \[ \binom n k = \binom n {n-k} \]

    \item Factoring in:

    \[ \binom n k = \frac n k \binom {n-1} {k-1} \]

    \item Sum over $k$:

    \[ \sum_{k = 0}^n \binom n k = 2 ^ n \]

    \item Sum over $n$:

    \[ \sum_{m = 0}^n \binom m k = \binom {n + 1} {k + 1} \]

    \item Sum over $n$ and $k$:

    \[ \sum_{k = 0}^m  \binom {n + k} k = \binom {n + m + 1} m \]

    \item Sum of the squares:

    \[ {\binom n 0}^2 + {\binom n 1}^2 + \cdots + {\binom n n}^2 = \binom {2n} n \]

    \item Weighted sum:

    \[ 1 \binom n 1 + 2 \binom n 2 + \cdots + n \binom n n = n 2^{n-1} \]

    \item Connection with the [Fibonacci numbers](../algebra/fibonacci-numbers.md):

    \[ \binom n 0 + \binom {n-1} 1 + \cdots + \binom {n-k} k + \cdots + \binom 0 n = F_{n+1} \]

\end{itemize}

\subsection{4 fundamental problems of distribution}

\subsubsection{$N$ equal balls in $K$ equal boxes}

\subsubsection{$N$ equal balls in $K$ distinct boxes}

Equivalent to count the number of solutions for the equation: 

\begin{equation}
    \begin{array}{c}
        x_1 + \ldots + x_K = N \\
        \text{where $x_i > 0$,  $i \in \left[1,K\right]$, N > 0}
    \end{array}
\end{equation}

It's given by the formula:

\begin{equation}
    \begin{array}{c}
        \binom{N+1}{K-1}
    \end{array}
\end{equation}

If some boxes may be \textbf{empty} ($xi \geq 0$), then it's given by:

$$
    \binom{N + K - 1}{N}
$$

\subsubsection{$N$ distinct balls in $K$ equal boxes}

\subsubsection{$N$ distinct balls in $K$ distinct boxes}

