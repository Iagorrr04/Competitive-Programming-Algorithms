\section{Number Theory}

\subsection{Fermat's Theorems and Lemmas}

Let $p$ be a prime number and $a, b \in \mathbb{Z}$:

\begin{equation}
a^p \equiv a \quad (\text{mod } p)
\end{equation}

\begin{equation}
a^{p-1} \equiv 1 \quad (\text{mod } p)
\end{equation}

\begin{equation}
  (a+b)^{p} \equiv a^{p} + b^{p} \quad (\text{mod } p)
\end{equation}

\begin{equation}
  a^{-1} \equiv a^{p-2} \quad (\text{mod } p)
\end{equation}




\subsection{Goldbach's Conjecture}

"Every pair number greater than 2 can be written as the sum of two primes"

Valid for every integer in range from $4$ to $10^{18}$, but without proof

For an odd $x$ number it can be written as the sum of two primes if $x - 2$ is also prime, or three primes, $3$ and the two primes that results in $x-3$.


\subsection{Linear Diophantine Equations}

A Linear Diophantine Equation (in two variables) is an equation of the general form:
 
$$ax + by = c$$ 

Where  $a$ ,  $b$ ,  $c$  are given integers, and  $x$ ,  $y$  are unknown integers.

If $ a = b = 0 $, we have infinite solutions if $ c = 0 $, and $0$ otherwise. 

\subsubsection{Solution(s)}

Let $g = gcd(a,b)$ such that $a x_g + b y_g = g$, then we only have a solution if and only if $g \mid c$, and if it have a solution it have infinite.

The solutions will be of the form :
 
\begin{equation}
  \begin{array}{c}
    x_0 = x_g \cdot \frac{c}{g},
 
    y_0 = y_g \cdot \frac{c}{g}. \\

    a \cdot x_0  + b \cdot y_0 = c 
  \end{array}
\end{equation}

With the initial solution, we can can find every solution, with : 

\begin{equation}
  \begin{array}{c}
    x = x_0 + k \cdot \frac{b}{g}, y = y_0 - k \cdot \frac{a}{g}
  \end{array}
\end{equation}

To find the solution that minimize $x + y$ we use the fact that: 

$$x' = x + k \cdot \frac{b}{g},$$ 
 
$$y' = y - k \cdot \frac{a}{g}.$$ 

Note that  

$x + y$  change as follows:
 
$$x' + y' = x + y + k \cdot \left(\frac{b}{g} - \frac{a}{g}\right) = x + y + k \cdot \frac{b-a}{g}$$ 

If  
$a < b$ , we need to select smallest possible value of  $k$ . If  $a > b$ , we need to select the largest possible value of  $k$ . If  $a = b$ , all solution will have the same sum  $x + y$ .


\subsection{Fundamental theorem of arithmetic}

Every integer greater than 1 can be represented uniquely as a product of prime numbers, up to the order of the factors.

\begin{equation}
    n = p_1^{\alpha_1} p_2^{\alpha_2} ... p_k^{\alpha_k}
\end{equation}

\subsubsection{LCM and GCD}

\begin{equation}
\begin{aligned}
    & a = p_1^{\alpha_1} p_2^{\alpha_2} ... p_k^{\alpha_k}  \\
    & b = p_1^{\beta_1} p_2^{\beta_2} ... p_k^{\beta_k} \\
    & (a,b) = p_1^{\min{{\alpha_1, \beta_1}}} p_2^{\min{{\alpha_2, \beta_2}}} ... p_k^{\min{{\alpha_k, \beta_k}}} \\
    & [a,b] = p_1^{\max{{\alpha_1, \beta_1}}} p_2^{\max{{\alpha_2, \beta_2}}} ... p_k^{\max{{\alpha_k, \beta_k}}}
\end{aligned}
\end{equation}


\subsection{Taking modulo at the exponent}

If $gcd(a,m) = 1$ then:

\begin{equation}
    a^{m} \equiv a^{n \mod \varphi(m)} \pmod{m}
\end{equation}
