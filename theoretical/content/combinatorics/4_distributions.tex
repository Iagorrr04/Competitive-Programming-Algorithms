\subsection{4 fundamental problems of distribution}

\subsubsection{$N$ equal balls in $K$ equal boxes}

Considering that no box can be empty, it's given by the partition function when the number of terms is limited by $k$.

\begin{equation}
  P(n,k)=P(n-1,k-1)+P(n-k,k)
\end{equation}

\subsubsection{$N$ equal balls in $K$ distinct boxes}

Equivalent to count the number of solutions for the equation: 

\begin{equation}
    \begin{array}{c}
        x_1 + \ldots + x_K = N \\
        \text{where $x_i > 0$,  $i \in \left[1,K\right]$, N > 0}
    \end{array}
\end{equation}

It's given by the formula:

\begin{equation}
    \begin{array}{c}
        \binom{N+1}{K-1}
    \end{array}
\end{equation}

If some boxes may be \textbf{empty} ($xi \geq 0$), then it's given by:

$$
    \binom{N + K - 1}{N}
$$

\subsubsection{$N$ distinct balls in $K$ equal boxes (Stirling's Number of Second Kind)}

Also known as \textbf{Stirling's Number of Second Kind} let's define $S(N,K)$ as how many distinct ways to distribute $N$ distinguishable balls in $K$ indistinguishable boxes.

Some special cases:


\begin{equation}
    \begin{array}{l}

        S(0,0)=1 \\
        S(n,0)=0 \\
        S(n,k)=0 \text{ if }n<k \\
        S(n,1)=1 \\
        S(n,k)=1 \text{ if } n=k \\
        S(n,2)=2^{n-1}-1  \\
        S(n,3)=\frac{1}{2}(3^{n-1}+1)-2^{n-1}  \\
        S(n,n-1)=\dbinom{n}{2} \\
        \displaystyle{S(n,n-2)=\dbinom{n}{3}+3 \dbinom{n}{4}} \\

    \end{array}
\end{equation}

Can be found by the following recurrence relationship.

\begin{equation}
    S(n,k)=S(n-1,k-1)+kS(n-1,k)
\end{equation}

There is also this formula :

\begin{equation}
    S(n,k)=\displaystyle{\sum_{r=0}^k (-1)^r \dfrac{(k-r)^n}{r!(k-r)!}}
\end{equation}

\begin{table}[h]
    \centering
    \begin{tabular}{l|llllllllll}
        \toprule
        \diagbox{N}{K}& 0 & 1 & 2 & 3 & 4 & 5 & 6 & 7 & 8 & 9 \\
        \midrule
        0 & 1 & 1 & 0 & 0 & 0 & 0 & 0 & 0 & 0 & 0 \\
        1 & 0 & 1 & 0 & 0 & 0 & 0 & 0 & 0 & 0 & 0 \\
        2 & 0 & 1 & 1 & 0 & 0 & 0 & 0 & 0 & 0 & 0 \\
        3 & 0 & 1 & 3 & 1 & 0 & 0 & 0 & 0 & 0 & 0 \\
        4 & 0 & 1 & 7 & 6 & 1 & 0 & 0 & 0 & 0 & 0 \\
        5 & 0 & 1 & 15 & 25 & 10 & 1 & 0 & 0 & 0 & 0 \\
        6 & 0 & 1 & 31 & 90 & 65 & 15 & 1 & 0 & 0 & 0 \\
        7 & 0 & 1 & 63 & 301 & 350 & 140 & 21 & 1 & 0 & 0 \\
        8 & 0 & 1 & 127 & 966 & 1701 & 1050 & 266 & 28 & 1 & 0 \\
        9 & 0 & 1 & 255 & 3025 & 7770 & 6951 & 2646 & 462 & 36 & 1 \\
        \bottomrule
    \end{tabular}
\end{table}


\subsubsection{$N$ distinct balls in $K$ distinct boxes}


